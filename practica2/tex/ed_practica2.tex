\documentclass[a4paper, 12pt]{article}
\usepackage[catalan]{babel}
\usepackage{hyperref}
\usepackage{fancyhdr}
\usepackage{graphicx}
\usepackage{tocloft}
\usepackage{listingsutf8}

\lstset{
    tabsize=2,
    language=Java, 
    breakatwhitespace=true,
    breaklines=true,
    numbers=left,
    frame=single,
    showspaces=false,
    showstringspaces=false,
    basicstyle=\small,
}

\begin{document}
\pagenumbering{gobble}
\begin{titlepage}

\raisebox{-0.2em}[0pt][0pt]{%
  \makebox[\textwidth][l]{\hbox{\hspace{-3.45em}\includegraphics[width=12em]{assets/logos/logo-campus.png}}}
}

\raisebox{1.9em}[0pt][0pt]{%
  \makebox[\textwidth][r]{\includegraphics[width=9em]{assets/logos/logo-udl.png}}
}

    \centering
    {\Large
    Universitat de Lleida\par
    Grau en Tècniques d'Interacció Digital i de Computació\par
    Estructura de Dades\par
}
    \vfill
    {\Huge\bfseries Pràctica 2\par}\vspace{3ex}
    {\Large\bfseries \textit{Programación Orientada a Objetos (POO) para las Estructuras de Datos}\par}
\vspace{6ex}
{\Large
Cristian Oprea i Xavier Vila
\par}
\vfill
\parbox{.5\textwidth}{\centering 22 d'octubre de 2020}
\end{titlepage}

\pagenumbering{arabic}
\section*{Ejercicio 1 - Aliases}
El nom en "x" es Marc\\
El nom en "y" es Marc\\
El nom en "z" es Joan\\
Això passa perquè l'objecte "x" apunta cap al objecte "y" degut al "x = y", per tant quan modifiquem l'atribut nom d'x realment estem modificant l'atribut d'on apunta "x", que és "y".
Cara a "z", com que no l'hem inicialitzat quan fem el "z = x" no estem creant cap apuntador, per tant el valor nom de "z" es manté igual.
\section*{Ejercicio 2 - Encapsulamiento}
a) Només pot accedir el bloc 2 degut a que \texttt{nameDay} no és una variable, és un atribut dels objectes de la classe "Ej2", per tant necessitem un objecte "Ej2" per poder accedir a aquest atribut.\\\\
b) No, no es pot imprimir perque l'atribut "x" és privat.\\\\
c) Per a solucionar això hem implementat un getter, per d'aquesta manera poder obtenir el valor de l'atribut "x" sense haver de fer la classe pública.
\begin{lstlisting}
package unPaquet;
public class A {
    private int x;
    public A() {
        x=1;
    }
    public int getX() {
        return x;
    }
}
package unPaquet;
public class C {
    A a;
    public C(){
        a=new A();
    }
    public void metodeC(){
        System.out.println("El valor de a es:" + a.getX());
    }
}
\end{lstlisting}

\end{document}
