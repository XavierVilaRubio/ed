\documentclass[a4paper, 12pt]{article}
\usepackage[catalan]{babel}
\usepackage{hyperref}
\usepackage{fancyhdr}
\usepackage{graphicx}
\usepackage{tocloft}
\usepackage{listingsutf8}

\begin{document}
\pagenumbering{gobble}
\begin{titlepage}

\raisebox{-0.2em}[0pt][0pt]{%
  \makebox[\textwidth][l]{\hbox{\hspace{-3.45em}\includegraphics[width=12em]{assets/logos/logo-campus.png}}}
}

\raisebox{1.9em}[0pt][0pt]{%
  \makebox[\textwidth][r]{\includegraphics[width=9em]{assets/logos/logo-udl.png}}
}

    \centering
    {\Large
    Universitat de Lleida\par
    Grau en Tècniques d'Interacció Digital i de Computació\par
    Estructura de Dades\par
}
    \vfill
    {\Huge\bfseries Pràctica 3\par}\vspace{3ex}
    {\Large\bfseries \textit{El tokenizer}\par}
\vspace{6ex}
{\Large
Cristian Oprea i Xavier Vila
\par}
\vfill
\parbox{.5\textwidth}{\centering 05 de novembre de 2020}
\end{titlepage}

\pagenumbering{arabic}
\section{L'estrategia}
Per a resoldre aquesta pràctica vam partir de la següent premisa: "Si obrim un parentesis, no podem tencar una clau fins que no tanquem el parentesis, i al revés".

Amb la premisa clara vam utilitzar un \textit{Iterator} per a anar recorrent l'array de \textit{Symbols} i, com podem obrir tants parentesis i claus com volguem, sempre que ens trobem un \textit{OpenParentesis} o bé un \textit{OpenLlave} l'afegirem a la pila.
Quan ens trobem un \textit{CloseParentesis} haurem de mirar si l'últim objecte afegit a la pila (és a dir, el "top") és un \textit{OpenParentesis}, ja que només podem tancar un parentesis si és l'últim que hem obert.
Si és així, fem un \textit{stack.pop();} i ens obvlidem d'aquell parentesis.
En cas de que no, fem un \textit{return false;} ja que si un parentesis tanca una clau l'expressió no seria correcte.

Finalment si no hi ha hagut cap incongruencia, fem un \textit{return stack.isEmpty();} per assegurar-nos que la pila està buida, ja que d'una altra manera l'expressió no seria correcte.

\end{document}
